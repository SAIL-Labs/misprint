%
%% This LaTeX was auto-generated from MATLAB code.
%% To make changes, update the MATLAB code and republish this document.
%
%\documentclass{article}
%\usepackage{graphicx}
%\usepackage{color}
%
%\sloppy
%\definecolor{lightgray}{gray}{0.5}
%\setlength{\parindent}{0pt}
%
%\begin{document}
%
%    
%    
\subsection*{Contents}

\begin{itemize}
\setlength{\itemsep}{-1ex}
   \item parse inputs
   \item start matlabpool if parallel computing tool box avaliable
   \item root path
   \item main reduction target path construction
   \item reference target path construction
   \item 1D spectra filenames
   \item check for required cards in fits header and read the values. add defaults where unavaliable.
   \item load misprint, and orientate so echelle dispersion is horizontal
   \item trace col
   \item load wavelength solution if supplied
   \item inital setup
   \item load data into local variables
   \item find orders
   \item trace orders
   \item scattered light estimate
\end{itemize}
\begin{verbatim}
classdef misprint < handleAllHidden
    %MISPRINT MultIorder SPectroscopic ReductIoN Tool
    %
    % Properties are set using key value pairs.
    %
    % Sample Usage:
    %     s2r = misprint('sciencespectrum','reference','flatspectrum','plotAlot',true,...
    %         'usecurrentfolderonly',true,...
    %         'numOfOrders',14,'numOfFibers',29,...
    %         'forceTrace',false,'forceExtract',false,...
    %         'forceDefineMaskEdge',false,'needsMask',false,...
    %         'peakcut',0.07,'minPeakSeperation',3,...
    %         'numTraceCol',40,'firstCol',140,'lastCol',300,...
    %         'parallel',false);
    %
    %     self.getMaskForIncompleteOrders;
    %     self.traceSpectra;
    %     self.extractSpectra;
    %     self.getP2PVariationsAndBlaze(false);
    %
    %     self.plotSpectraFor(1:14,true,false)
    %
    % Copyright (C) Chris Betters 2012-2014

    properties
        targetBaseFilename, % base filename of target (file with spectra to be extracted) fits file.
        targetPath, % path to target fits file.
        rootDirectory, % path to current directory, should equal [pwd '/'].
        referenceBaseFilename, % base filename of reference fits file (i.e. a flat frame).
        referencePath, % path to reference fits file.

        targetHeader, % structure with target fits header.
        referenceHeader, % structure with reference fits header.

        SpectraFitsSaveFileName, % filename of fits file to save extracted spectra to.
        ReferenceSpectraFitsSaveFileName, % spectra previously extracted from the reference fits file.
        FlatReferenceSpectraFitsSaveFileName, % filename of fits file to for use as a flat reference, should be just pixel to pixel variations.

        spectraTracePath, % path to previously saved trace data for reference fits file.

        useReference, % flag to indicate if valid reference data has been set/ffound.
        plotAlot, % flag to show raw image and plots during tracing. It can plot alot.
        forceTrace, % flag to force a trace of spectra in the current image. Can not be set with a reference.
        forceExtract, % flag to force a new extraction of the current image. This will overwrite the default save file if it exists.
        forceDefineMaskEdge, % flag to force a new definition of the mask/clipping region.
        needsMask, % flag to indicate if image requires clipping.
        clipping, % vector of pixels from [left top right bottom] to clip.
        parallel, % flag tin indicate if parallel compuyting toolbox should/can be used.
        minPeakSeperation, % min peak seperation for tracing and peakfinder

        numOfOrders, % number of diffraction orders in image.
        numOfFibers, % number of spectra (fibres) in each order.

        gain, % gain (e-/adu) read from fits file
        readNoise, % read noise (rms e-) read from fits file
        dispAxis, % axis of primary dispersion read from fits file

        imdata, % target image data.
        imvariance, % estiamted variance for target image data.
        mask, % mask of clipped regions.
        imdim, % size of imdata (equals size(imdata))

        spectraValues, % extracted spectra values.
        spectraVar, % var for extracted spec values.
        backgroundValues, % background value from extraction

        finalSpectra, %  linearised version of complete spectrum for individual fibre (all orders)
        finalSpec, % linearised combined spectrum
        finalSpectraVar, % varience for finalSpectra
        finalSpecVar, % varience for finalSpec
        finalWave, % linearised wavelength scale for finalSpectra and finalSpec

        referenceSpectraValues, % extracted spectraValues of reference/flat
        P2PVariationValues, % pixel to pixel variation from reference.
        flatBlaze, % estimated blaze from reference.

        wavematfile, % mat file with wavelength fit paramaters
        wavefit,% wavelength soultion for each fibre
        diffractionOrder, % estimated diffraction order from wavefit

        numTraceCol, % number of columns to use when tracing.
        firstCol, % first column of trace
        lastCol, % last column of trace

        orderEdges,  % detected edges of the orders
        specCenters, % polynoimail interpolated fitted y axis centeres of the spectra from gaussain fit.
        specWidth,   % polynomial interpolated width of the spectra from gaussain fit.
        meanSpecWidth, % mean width of spceetra in each order.
        meanOrderWidth, % mean width of each order

        fittedCenters, % center of spectrum (vertical) from gaussian fit
        fittedCol, % column used to get progile for fit
        fittedWidth, % width of specturm (vertical) from gaussain fit
        fittedParamters, % all fit paramaters from trace

        usecurrentfolderonly, % flag to note use my maxiumDL/PIMMS echelle file structure.
        peakcut, % MINPEAKHEIGHT for spectra tracing detection. (fraction of mean of current profile).

        OXmethod, % name of method to use for optimal extraction.
    end

    methods
        function self=misprint(targetBaseFilename,varargin)
\end{verbatim}
\begin{verbatim}
            % init the MISPRINT class. Pasre all inputs, load relevant files.
\end{verbatim}


\subsection*{parse inputs}

\begin{verbatim}
            p = inputParser;

            p.addRequired('targetBaseFilename', @(x) ischar(x));
            p.addParamValue('reference','');
            p.addParamValue('forceTrace'            , false, @(x) islogical(x));
            p.addParamValue('forceExtract'          , false, @(x) islogical(x));
            p.addParamValue('plotAlot'      , false, @(x) islogical(x));
            p.addParamValue('forceDefineMaskEdge', false, @(x) islogical(x));
            p.addParamValue('needsMask', true, @(x) islogical(x));
            p.addParamValue('numOfOrders', 15, @(x) isnumeric(x));
            p.addParamValue('numOfFibers', 19, @(x) isnumeric(x));
            p.addParamValue('usecurrentfolderonly',false, @(x) islogical(x));
            p.addParamValue('peakcut', 0.8, @(x) isnumeric(x));
            p.addParamValue('parallel',true, @(x) islogical(x));
            p.addParamValue('numTraceCol', 10, @(x) isnumeric(x));
            p.addParamValue('dispAxis', [], @(x) isnumeric(x));
            p.addParamValue('wavesolution', '', @(x) ischar(x));
            p.addParamValue('minPeakSeperation', 3, @(x) isnumeric(x));
            p.addParamValue('firstCol', 0, @(x) isnumeric(x));
            p.addParamValue('lastCol', 0, @(x) isnumeric(x));
            p.addParamValue('clipping',[0 0 0 0], @(x) isnumeric(x) && length(x)==4)
            p.addParamValue('OXmethod','MPDoptimalExtBack', @(x) ismethod(self,x))

            p.parse(targetBaseFilename,varargin{:});

            self.numOfOrders=p.Results.numOfOrders;
            self.numOfFibers=p.Results.numOfFibers;
            self.forceExtract=p.Results.forceExtract;
            self.forceTrace=p.Results.forceTrace;
            self.plotAlot=p.Results.plotAlot;
            self.forceDefineMaskEdge=p.Results.forceDefineMaskEdge;
            self.needsMask=p.Results.needsMask;

            self.usecurrentfolderonly=p.Results.usecurrentfolderonly;
            self.peakcut=p.Results.peakcut;

            self.parallel=p.Results.parallel;
            self.numTraceCol=p.Results.numTraceCol;
            self.firstCol=p.Results.firstCol; %if zero set to 20 minus miage size (at end)
            self.lastCol=p.Results.lastCol; %if zero set to 20 minus miage size (at end of init)

            self.dispAxis=p.Results.dispAxis;

            self.minPeakSeperation=p.Results.minPeakSeperation;

            self.clipping=p.Results.clipping;
            self.OXmethod=p.Results.OXmethod;
\end{verbatim}

        \color{lightgray} \begin{verbatim}Error using chrislib.misprint (line 130)
Not enough input arguments.\end{verbatim} \color{black}
    

\subsection*{start matlabpool if parallel computing tool box avaliable}

\begin{verbatim}
            if self.parallel
                if license('test', 'distrib_computing_toolbox')
                    if isempty(gcp('nocreate'))
                        parpool('local')
                    end
                else
                    warning('MISPRINT:init:useDistribComputingToolbox:notAvalaible','The parallel computing toolbox is not avaliable, but has been requested.')
                end
            end
\end{verbatim}


\subsection*{root path}

\begin{verbatim}
            self.rootDirectory=[pwd '/'];
\end{verbatim}


\subsection*{main reduction target path construction}

\begin{verbatim}
            self.targetBaseFilename=p.Results.targetBaseFilename;

            if ~self.usecurrentfolderonly
                self.targetPath = [self.rootDirectory self.targetBaseFilename '/reduced/' self.targetBaseFilename '-master.fit'];
            else
                self.targetPath = [self.rootDirectory self.targetBaseFilename '.fit'];
            end
            % check the file is a valid fits.
            self.checkForReducedFitsAt(self.targetPath);

            % get fits header
            self.targetHeader=fitsheader(self.targetPath);
\end{verbatim}


\subsection*{reference target path construction}

\begin{verbatim}
            self.referenceBaseFilename=p.Results.reference;

            if isempty(self.referenceBaseFilename)
                self.useReference     = false;
                self.spectraTracePath = [self.rootDirectory self.targetBaseFilename '-trace.mat'];
            else
                self.useReference     = true;
                self.spectraTracePath = [self.rootDirectory self.referenceBaseFilename '-trace.mat'];
                if ~self.usecurrentfolderonly
                    self.referencePath    = [self.rootDirectory self.referenceBaseFilename '/reduced/' self.referenceBaseFilename '-master.fit'];
                else
                    self.referencePath    = [self.rootDirectory self.referenceBaseFilename '.fit'];
                end
                % check the file is a valid fits.
                self.checkForReducedFitsAt(self.targetPath);

                % get fits header
                self.referenceHeader  = fitsheader(self.referencePath);


                self.ReferenceSpectraFitsSaveFileName=[self.referenceBaseFilename '-1D-spectra.fits'];
                self.referenceSpectraValues=fitsread(self.ReferenceSpectraFitsSaveFileName);

                self.FlatReferenceSpectraFitsSaveFileName=[self.referenceBaseFilename '-flattend-1D-spectra.fits'];

                %assertWarn(isfield(self.referenceHeader,'IMAGETYP') && strcmp(self.referenceHeader.IMAGETYP,'Flat Frame'),...
                %    'MISPRINT:init:referenceNotAFlat',...
                %    'Reference Frame has not been tagged as a flat in fits header')
            end
\end{verbatim}


\subsection*{1D spectra filenames}

\begin{verbatim}
            self.SpectraFitsSaveFileName=[self.targetBaseFilename '-1D-spectra.fits'];
\end{verbatim}


\subsection*{check for required cards in fits header and read the values. add defaults where unavaliable.}

\begin{verbatim}
            if isfield(self.targetHeader,'READNOIS')
                self.readNoise=self.targetHeader.READNOIS;
            elseif isfield(self.targetHeader,'RO_NOISE')
                self.readNoise=self.targetHeader.RO_NOISE;
            else
                self.readNoise=11.3; % atik default
                %                 if strcmp(self.targetHeader.INSTRUME,'ArtemisHSC')
                %                     fitsAddHeaderKeyword(self.targetPath,'READNOIS',self.readNoise,' ')
                %                 end
            end

            if isfield(self.targetHeader,'GAIN')
                self.gain=self.targetHeader.GAIN;
            elseif isfield(self.targetHeader,'RO_GAIN')
                self.gain=self.targetHeader.RO_GAIN;
            else
                self.gain=0.43; % fli default
                %                 if strcmp(self.targetHeader.INSTRUME,'ArtemisHSC')
                %                     fitsAddHeaderKeyword(self.targetPath,'GAIN',self.gain,' ')
                %                 end
            end

            if isempty(self.dispAxis)
                if isfield(self.targetHeader,'DISPAXIS')
                    self.dispAxis=self.targetHeader.DISPAXIS;
                else
                    self.dispAxis=1; % atik default
                    %if strcmp(self.targetHeader.INSTRUME,'ArtemisHSC')
                    %    fitsAddHeaderKeyword(self.targetPath,'DISPAXIS',self.dispAxis,' ');
                    %end
                end
            end
\end{verbatim}


\subsection*{load misprint, and orientate so echelle dispersion is horizontal}

\begin{verbatim}
            self.imdata=fitsread(self.targetPath);

            if self.dispAxis==1
                self.imdata=fliplr(self.imdata'); %
            end

            if sum(self.clipping)
                %[left top right bottom]
                self.imdata=self.imdata(max([1 self.clipping(2)]):end-self.clipping(4),max([1 self.clipping(1)]):end-self.clipping(3));
                if ~isempty(self.wavefit)
                    self.wavefit=self.wavefit(:,max([1 self.clipping(1)]):end-self.clipping(3),:);
                end
            end

            %self.imdata=rot90(self.imdata,2);
            self.imdim=size(self.imdata);
            self.imvariance=(self.readNoise/self.gain)^2 + abs(self.imdata) / self.gain; %http://cxc.cfa.harvard.edu/mta/ASPECT/aca_read_noise/
\end{verbatim}


\subsection*{trace col}

\begin{verbatim}
            if ~self.lastCol
                self.lastCol=self.imdim(2)-20;
            end
            if ~self.firstCol
                self.firstCol=20;
            end
\end{verbatim}


\subsection*{load wavelength solution if supplied}

\begin{verbatim}
            if ~isempty(p.Results.wavesolution)
                self.wavematfile=p.Results.wavesolution;
                wavepayload=load(self.wavematfile,'p','S','mu');
                p=wavepayload.p;
                S=wavepayload.S;
                mu=wavepayload.mu;
                self.wavefit=zeros(self.numOfFibers,self.imdim(2),self.numOfOrders);
                for o=1:self.numOfOrders;
                    for f=1:self.numOfFibers;
                        self.wavefit(f,:,o)=polyval(p(f,:,o),1:self.imdim(2),S(f,:,o),mu(f,:,o));
                    end
                end
                self.diffractionOrder=round(2*1e-3/31.6*cosd(5)*sind(63.2)./(mean(squeeze(mean(self.wavefit,2)),1)*1e-9)');
            end
\end{verbatim}
\begin{verbatim}
        end

        function runDefaultExtraction(self)
            % run default set of extraction commands

            self.getMaskForIncompleteOrders;
            self.traceSpectra;
            self.extractSpectra;
            self.getP2PVariationsAndBlaze
        end

        function traceSpectra(self,varargin)
\end{verbatim}
\begin{verbatim}
            % trace spectra from flat.
            %
            % optional inputs misprint.traceSpectra(inputimage,numOfOrders,numOfFibers)
            % inputimage is same format as misprint.imdata
\end{verbatim}


\subsection*{inital setup}

\begin{par}
if preexisting trace exists it is loaded (unless forceTrace set)
\end{par} \vspace{1em}
\begin{verbatim}
            if nargin==1
                if (~exist(self.spectraTracePath,'file') || self.forceTrace ) && ~self.useReference

                    assertWarn(self.forceTrace & exist(self.spectraTracePath,'file'),...
                        'MISPRINT:traceSpectra:TraceForced',...
                        'Tracing was forced, this will overwrite previous trace.')

                    if ~exist(self.spectraTracePath,'file'); disp(['Tracefile: ' self.spectraTracePath ' does not exist.']);end
                else
                    assert(~(self.forceTrace & self.useReference),...
                        'MISPRINT:traceSpectra:TraceForcedWithUseReferenceSet',...
                        'Tracing can not be forced when useReference is set')

                    assert(~(self.useReference & ~exist(self.spectraTracePath,'file')),...
                        'MISPRINT:traceSpectra:ReferecnceTraceFileNotFound',...
                        [self.spectraTracePath ' was not found and is required as a reference. Aborting extraction.'])

                    load(self.spectraTracePath,'specCenters','specWidth','orderWidth','orderEdges','means','columns','widths','fitxs')

                    if self.useReference
                        %disp(['Using reference trace: ' self.spectraTracePath])
                    else
                        disp(['Using previous trace: ' self.spectraTracePath])
                    end

                    self.meanSpecWidth=squeeze(mean(specWidth,2));
                    self.meanOrderWidth=squeeze(mean(orderWidth,2));
                    self.specCenters=specCenters;
                    self.specWidth=specWidth;
                    self.orderEdges=orderEdges;

                    self.fittedCenters=means;
                    self.fittedCol=columns;
                    self.fittedWidth=widths;
                    self.fittedParamters=fitxs;

                    return % end function call after loading data
                end
            end
\end{verbatim}


\subsection*{load data into local variables}

\begin{verbatim}
            x=1:self.imdim(1);
            if nargin==4
                inputimage = varargin{1};
                numOfOrders = varargin{2};
                numOfFibers = varargin{3};
            else
                inputimage=self.imdata;
                numOfOrders = self.numOfOrders;
                numOfFibers = self.numOfFibers;
            end

            imdata=inputimage.*self.mask;
\end{verbatim}


\subsection*{find orders}

\begin{verbatim}
            if (self.numTraceCol>=self.imdim(2))
                columns=1:self.imdim(2);
                warning('MISPRINT:fitAllOfTheThings','You just asked for a fit to every column.')
                reply = input('Are your sure?? Y/N [Y]: ', 's');
                if ~strcmpi('Y',reply)
                    error('MISPRINT:traceSpectra:userInterpupt','MISPRINT termintated in traceSpectra.')
                end
            else
                columns=round(linspace(self.firstCol, self.lastCol, self.numTraceCol));
            end

            imcol=imdata(:,columns); % sliced image

            disp('Running order tracer. This may take some time.')
            for i=1:length(columns)
                [yp,index]=findpeaks(imcol(:,i),'NPEAKS',numOfOrders*numOfFibers,'MINPEAKHEIGHT',max(imcol(:,i))*self.peakcut,'MINPEAKDISTANCE',self.minPeakSeperation);
                if self.plotAlot
                    figure(i);clf
                    plot(x,imcol(:,i),index,yp,'xr');
                    line([1 length(imcol(:,i))],[max(imcol(:,i)) max(imcol(:,i))]*self.peakcut)
                    title([num2str(columns(i))])
                end
                if numOfOrders==1
                    orderWidth=self.imdim(1);
                    orderCenter=round(self.imdim(1)/2);
                    orderEdges(:,i)=[1 self.imdim(1)];
                else
                    orderWidth=diff(index(1:numOfFibers:end));
                    orderCenter=mean([index(numOfFibers:numOfFibers:end) index(1:numOfFibers:end)],2);
                    %error(' ')
                    orderEdges(:,i)=[orderCenter(1)-orderWidth(1)/2;...
                        mean([index(numOfFibers:numOfFibers:end-numOfFibers)...
                        index(numOfFibers+1:numOfFibers:end)],2); orderCenter(end)+orderWidth(end)/2];
                end

                %                 if self.plotAlot
                %                     plot(imcol(:,i))
                %                     hold on
                %                     %line(repmat(orderEdges(:,i)',[2,1]),[zeros(1,size(orderEdges(:,i),2)); ones(1,size(orderEdges(:,i),2))*max(imcol)])
                %                     hold off
                %                 end
            end
\end{verbatim}


\subsection*{trace orders}

\begin{par}
fit gaussian to profile in columns for each order.
\end{par} \vspace{1em}
\begin{verbatim}
            fitxs=zeros(numOfOrders,3*numOfFibers+1,length(columns));
            for i=1:length(columns)
                for order=1:numOfOrders
                    disp(['Column:' num2str(columns(i)) ' | Fitting Spectra in Order: ' num2str(order)])

                    orderProfileX=round(max(orderEdges(order,i),1):min(orderEdges(order+1,i),self.imdim(1)))';
                    orderProfile=imcol(orderProfileX,i);
                    orderProfile=orderProfile/max(orderProfile);


                    [~, means(order,:,i), widths(order,:,i), fitxs(order,:,i)] = ...
                        self.fitNGaussainsAlt(numOfFibers,orderProfileX, orderProfile,self.peakcut);

                    %                    [~, means(order,:,i), widths(order,:,i), fitxs(order,:,i)] = ...
                    %                         fitNGaussains(numOfFibers,orderProfileX, orderProfile,self.peakcut,false);

                    if self.plotAlot
                        figure(i);clf;
                        %subplot(5,4,columns)
                        plot(orderProfileX,sum(self.nGausFunc(fitxs(order,:,i),orderProfileX,numOfFibers),2),'r-',...
                            orderProfileX,orderProfile,'-')
                        title(['Order: ' num2str(order) ' Column: ' num2str(columns(i))])
                        %pause(0.1)
                    end
                end
            end

            specCenters=self.polyfitwork(self.imdim,means,columns,3);
            specWidth=self.polyfitwork(self.imdim,widths,columns,3);
            meanSpecWidth=squeeze(mean(self.specWidth,3));

            self.fittedCenters=means;
            self.fittedCol=columns;
            self.fittedWidth=widths;

            self.meanSpecWidth=meanSpecWidth;
            self.specCenters=specCenters;
            self.specWidth=specWidth;
            self.orderEdges=orderEdges;
            self.fittedParamters=fitxs;
            self.meanOrderWidth=squeeze(mean(orderWidth,2));

            save(self.spectraTracePath,'specCenters','specWidth','orderWidth','orderEdges','means','columns','widths','fitxs')
\end{verbatim}
\begin{verbatim}
        end

        function getMaskForIncompleteOrders(self)
            %  get mask for incomplete orders

            if ~self.needsMask
                self.mask=ones(self.imdim);
                return % no clipe, so mask is ones.
            end

            if isfield(self.targetHeader,'CLIPTL') && isfield(self.targetHeader,'CLIPTR') && isfield(self.targetHeader,'CLIPBL') && isfield(self.targetHeader,'CLIPBR')
                topEdges=[self.targetHeader.CLIPTL self.targetHeader.CLIPTR];
                bottomEdges=[self.targetHeader.CLIPBL self.targetHeader.CLIPBR];
            else
                self.forceDefineMaskEdge=true; % overide default as clipping is needed, and data not defined
            end

            if self.forceDefineMaskEdge
                echfig=figure(1);
                imagesc(self.imdata);
                axis([1 self.imdim(2) 1 self.imdim(1)*0.5]) % show top half of image
                [~,y]=getpts(echfig);
                topEdges=[y(1) y(2)];
                axis([1 self.imdim(2) self.imdim(1)-self.imdim(1)*0.3 self.imdim(1)]) % show bottom third of image
                [~,y]=getpts(echfig);
                bottomEdges=[y(1) y(2)];
            end

            % make mask of image to exclude incomplete orders
            xi=[0; self.imdim(2);              self.imdim(2);                  0];
            yi=[0;        0;    topEdges(2); topEdges(1)];
            BW1 = roipoly(self.imdata,xi,yi);

            xi=[       0; self.imdim(2);           self.imdim(2);                  0];
            yi=[self.imdim(1); self.imdim(1);     bottomEdges(2);     bottomEdges(1)];

            BW2 = roipoly(self.imdata,xi,yi);
            self.mask=~BW1 & ~BW2;

            if self.forceDefineMaskEdge
                imagesc(self.imdata.*self.mask)
                reply = input('Should I add to Fits Header Y/N [N]: ', 's');
                if isempty(reply)
                    reply = 'N';
                end
                if strncmpi(reply,'Y',1)
                    disp('saving clips to header')
                    fitsAddHeaderKeyword(self.targetPath,'CLIPTL',topEdges(1),' ');
                    fitsAddHeaderKeyword(self.targetPath,'CLIPTR',topEdges(2),' ');
                    fitsAddHeaderKeyword(self.targetPath,'CLIPBL',bottomEdges(1),' ');
                    fitsAddHeaderKeyword(self.targetPath,'CLIPBR',bottomEdges(2),' ');
                end
            end

            if self.plotAlot
                figure(1)
                imagesc(self.imdata.*self.mask)
            end

        end

        function getP2PVariationsAndBlaze(self,varargin)
            % get smoothed version of flat spectrum (ie blaze) and pixel to
            % pixel variations (flatspectrum./smooth flat spectrum)
            %
            %  load reference
            if length(varargin)==2
                referenceFile=varargin{2};
            elseif self.useReference
                referenceFile=self.ReferenceSpectraFitsSaveFileName;
            end

            if ~isempty(varargin)
                force=varargin{1};
            else
                force=false;
            end

            matpayload=load(self.spectraTracePath,'flatBlaze','P2PVariationValues');
            if ~force && isfield(matpayload,'flatBlaze') && isfield(matpayload,'P2PVariationValues')
                self.flatBlaze=matpayload.flatBlaze;
                self.P2PVariationValues=matpayload.P2PVariationValues;
            else
                assertWarn(force,'MISPRINT:getP2PVariationsAndBlaze:forced','P2P and blaze forced')
                mask=ones(self.numOfFibers,self.imdim(2));
                %                 mask(end-50:end)=NaN;
                %                 mask(1:50)=NaN;

                spectraValues=self.spectraValues;
                %                 for or=1:self.numOfOrders
                %                     spectraValues(:,:,or)=bsxfun(@rdivide,spectraValues(:,:,or).*mask,max(spectraValues(:,:,or)')');
                %                 end
                flatBlaze=zeros(size(self.spectraValues));
                P2PVariationValues=zeros(size(self.spectraValues));

                for f=1:self.numOfFibers
                    for or=1:self.numOfOrders
                        %error('')
                        flatBlaze(f,:,or)=csaps(1:self.imdim(2),spectraValues(f,:,or),1e-5,1:self.imdim(2));
                        %flatBlaze(f,:,or)=smooth(spectraValues(f,:,or),100);
                    end
                end
                P2PVariationValues=spectraValues./flatBlaze;
                self.flatBlaze=flatBlaze;
                self.P2PVariationValues=P2PVariationValues;

                self.flatBlaze(isnan(self.flatBlaze))=1;
                self.P2PVariationValues(isnan(self.P2PVariationValues))=1;

                save(self.spectraTracePath,'flatBlaze','P2PVariationValues','-append')
            end
        end

        function getBlazeAlt(self,varargin)
            if ~isempty(varargin)
                force=varargin{1};
            else
                force=false;
            end

            matpayload=load(self.spectraTracePath,'flatBlaze','P2PVariationValues');
            if ~force && isfield(matpayload,'flatBlaze') && isfield(matpayload,'P2PVariationValues')
                self.flatBlaze=matpayload.flatBlaze;
                self.P2PVariationValues=matpayload.P2PVariationValues;
            else
\end{verbatim}
\begin{verbatim}
                assertWarn(force,'MISPRINT:getP2PVariationsAndBlaze:forced','P2P and blaze forced')
\end{verbatim}
\begin{verbatim}
                x=[1:2498];

                opts1 = fitoptions( 'Method', 'LinearLeastSquares' );
                opts1.Normalize = 'on';
                ft1=fittype('poly2');

                opts2 = fitoptions( 'Method', 'LinearLeastSquares' );
                opts2.Normalize = 'on';
                ft2=fittype('poly3');
\end{verbatim}
\begin{verbatim}
                for f=1:self.numOfFibers
                    for o=1:self.numOfOrders
                        spec=self.spectraValues(f,:,o);

                        % ft=fittype( 'smoothingspline' );
                        % opts = fitoptions( 'Method', 'SmoothingSpline' );
                        % opts.Normalize = 'on';
                        % opts.SmoothingParam = 1e-5;

                        [fitresult, gof] = fit( x', spec', ft1, opts1);
                        %plot(detrend(spec'./feval(fitresult,x))+1); grid on

                        ignore=detrend(spec'./feval(fitresult,x))+1 < 1;

                        [fitresult, gof] = fit( x(~ignore)', medfilt1(spec(~ignore),10)', ft2, opts2);

                        blaze=feval(fitresult,x);
                        flatBlaze(f,:,o)=blaze./max(blaze);
                        %plot(x,squeeze(flatBlaze(f,:,o)),x,self.spectraValues(f,:,o)/max(self.spectraValues(f,:,o)))
                        %pause(0.2)
                    end
                end

                P2PVariationValues=ones(size(flatBlaze));
                self.flatBlaze=flatBlaze;
                self.P2PVariationValues=P2PVariationValues;

                %save(self.spectraTracePath,'flatBlaze','P2PVariationValues','-append')
\end{verbatim}
\begin{verbatim}
            end
        end

        function plotSpectraValuesFor(self,orders,shouldFlat,shouldP2PV)
            % plot spectra orders specifed. three arguments orders,shouldFlat,shouldP2PV

            if shouldFlat && shouldP2PV
                FlattenedSpectra=self.spectraValues./self.flatBlaze./self.P2PVariationValues;
            elseif shouldFlat && ~shouldP2PV
                FlattenedSpectra=self.spectraValues./self.flatBlaze;
            elseif ~shouldFlat && shouldP2PV
                FlattenedSpectra=self.spectraValues./self.P2PVariationValues;
            else
                FlattenedSpectra=self.spectraValues;
            end

            if isempty(self.wavefit)
                for order=orders
                    figure(order);clf;
                    subplot(1,2,1)
                    %imagesc(log10(self.imdata-min2(self.imdata)+1))
                    imagesc(self.imdata)
                    %set(gca, 'CLim',[0 1000])
                    ylim([min2(squeeze(self.specCenters(order,:,:)))-50 max2(squeeze(self.specCenters(order,:,:)))+50])
                    hold on
                    plot(1:self.imdim(2),squeeze(self.specCenters(order,:,:)),'k','LineWidth',1.5)
                    %axis image

                    subplot(1,2,2)
                    for f=1:self.numOfFibers
                        FlattenedSpectraNorm(f,:,order)=FlattenedSpectra(f,:,order)/(max(FlattenedSpectra(f,:,order)));
                        plot(1:self.imdim(2),FlattenedSpectraNorm(f,:,order)+(self.numOfFibers-f)*0.2)
                        hold all
                    end
                end
                hold off
                grid on
            else
                for order=orders
                    figure(order);clf;

                    subplot(1,2,1)
                    imagesc(log10(self.imdata-min2(self.imdata)+1))
                    %imagesc(self.imdata)
                    %set(gca, 'CLim',[0 1000])
                    ylim([min2(squeeze(self.specCenters(order,:,:)))-50 max2(squeeze(self.specCenters(order,:,:)))+50])
                    hold on
                    plot(1:self.imdim(2),squeeze(self.specCenters(order,:,:)),'k','LineWidth',1.5)
                    title(['Order ' num2str(self.diffractionOrder(order))])
                    xlabel('Primary-dispersion axis (pixels)')
                    ylabel('Cross-dispersion axis (pixels)')

                    subplot(1,2,2)
                    for f=1:self.numOfFibers
                        FlattenedSpectraNorm(f,:,order)=FlattenedSpectra(f,:,order)/max(FlattenedSpectra(f,:,order));
                        plot(self.wavefit(f,:,order),FlattenedSpectraNorm(f,:,order)+(self.numOfFibers-f)*0.2)
                        hold all
                    end

                    %                     plot(self.wavefit(:,:,order)',FlattenedSpectra(:,:,order)')
                    %                     xlabel('Wavelength (nm)')
                end
                hold off
                grid on
            end
        end

        function plotSingleFibre(self,f,shouldFlat,shouldP2PV)
            % plot spectra for signle fibre across multiple orders, three arguments fibre,shouldFlat,shouldP2PV
            if shouldFlat && shouldP2PV
                FlattenedSpectra=self.spectraValues./self.flatBlaze./self.P2PVariationValues;
            elseif shouldFlat && ~shouldP2PV
                FlattenedSpectra=self.spectraValues./self.flatBlaze;
            elseif ~shouldFlat && shouldP2PV
                FlattenedSpectra=self.spectraValues./self.P2PVariationValues;
            else
                FlattenedSpectra=self.spectraValues;
            end

            if isempty(self.wavefit)
                plot(bsxfun(@plus, repmat([1:self.imdim(2)],[self.numOfOrders 1])',[0:self.numOfOrders-1]*self.imdim(2)),squeeze(FlattenedSpectra(f,:,:)))
            else
                plot(squeeze(self.wavefit(f,:,:)),squeeze(FlattenedSpectra(f,:,:)))
            end
        end

        function plotFinalSpectra(self)
            for f=1:self.numOfFibers
                plot(self.finalWave,self.finalSpectra(f,:)+(self.numOfFibers-f)*2)
                hold all
            end
            hold off
            xlabel('Wavelength (nm)')
            hold on
            orderwaveedges=squeeze(max(min(self.wavefit,[],2),[],1));
            line([orderwaveedges orderwaveedges], [0 40],'LineWidth',1,'Color','k')

            hold off
        end

        function plotFinalSpec(self)
            %             [sunflux, sunwave] = getsunspec(min(self.finalWave), max(self.finalWave), 0.022);
            %             [telflux, telwave] = getTelluricSpec(min(self.finalWave), max(self.finalWave), 0.022);
            %
            %             plot(sunwave,sunflux,telwave,telflux)
            %            hold all
            plot(self.finalWave,self.finalSpec/max(self.finalSpec))
            orderwaveedges=squeeze(max(min(self.wavefit,[],2),[],1));
            hold on
            line([orderwaveedges orderwaveedges], [0 1],'LineWidth',2,'Color','k')

            hold off
            xlabel('Wavelength (nm)')
        end

        function filterBadPixels(self,Nsigma,thresh,shouldPlot)
            % filter bad pixels. three arguments Nsigma,thresh,shouldPlot
            im=self.imdata;
            im(im<=0)=1;

            imdiff=medfilt2(im,[2 2])./im; % try and highlight odd pixels

            imdiff=imdiff-mean2(imdiff); % set mean to zero

            bad1=imdiff>thresh; % very larger value can bias std, so clip them.

            badpixel=(imdiff>std2(imdiff(~bad1))*Nsigma | imdiff<-std2(imdiff(~bad1))*Nsigma ) | bad1;

            self.imdata(badpixel)=NaN;

            self.imdata=inpaint_nans(self.imdata);

            if shouldPlot
                figure(shouldPlot);clf
                [badx, bady]=find(badpixel);

                imagesc(self.imdata)
                hold on
                shouldPlot(bady,badx,'wx')
                hold off
            end

        end

        function blurred=removeIntensityGradientInImdata(self,avgWin)
            % smooth whole image, then divided original by that. Usefull for to
            % improve flat tracing. one arguments avgWin (window for smoothing)
            PSF = fspecial('average', [1 1]*avgWin);

            blurred = imfilter(self.imdata, PSF, 'conv', 'symmetric');
            blurred=blurred/mean2(blurred);

            if self.plotAlot
                subplot(1,3,1)
                imagesc(self.imdata)
                subplot(1,3,2)
                imagesc(blurred)
                subplot(1,3,3)
                imagesc(self.imdata./blurred)
            end
            %self.imdata=self.imdata./blurred;
        end

        function getBackgroundBetweenOrders(self)
\end{verbatim}
\begin{verbatim}
            self.imdata(self.imdata<0)=0;
            locs=self.orderEdges';
            locs(locs>3362)=3362;

            imagesc(log10(self.imdata))
            hold on;
            plot(self.fittedCol,locs','bx')
            hold off;
            pks=[];


            filtedimdata=medfilt2(self.imdata);

            %locs(89,:)=(locs(88,:) + locs(90,:)) / 2;

            for i = 1:size(locs,2)
                %    pks(i,:)=self.imdata(self.fittedCol,round(self.orderEdges(,:)));
                p = impixel(filtedimdata,self.fittedCol,locs(:,i)');
                pks(:,i) = p(:,1);
            end



            cols=self.fittedCol;
\end{verbatim}


\subsection*{scattered light estimate}

\begin{verbatim}
            invertedimdata=(1./(self.imdata)).*self.mask;
            invertedimdata(isinf(invertedimdata))=0;
            figure(1)
            imagesc(log10(invertedimdata))
            x=1:self.imdim(1);


            inverpks=1./pks;
            cols2=repmat(cols,[size(locs,2),1])';
            figure(3);clf
            h(2)=surface(cols2,locs,pks,'EdgeColor','none');
            xlim([1 self.imdim(2)])
            ylim([1 self.imdim(1)])
            set(get(h(2),'Parent'),'YDir','reverse')


            figure(2);clf
            sfun=scateringTestFit(cols2, locs, inverpks);



            figure(4);clf
            [XI,YI]=meshgrid(1:self.imdim(2), 1:self.imdim(1));

            subplot(1,2,2)
            imagesc(1./feval(sfun,XI,YI).*self.mask)
            title('Estimated Scattering (from Inter-Order Regions)')

            subplot(1,2,1)
            h(2)=surface(cols2,locs,1./pks,'EdgeColor','none');
            xlim([1 self.imdim(2)])
            ylim([1 self.imdim(1)])
            set(get(h(2),'Parent'),'YDir','reverse')


            %self.imdata=self.imdata-feval(sfun,X,Y)
            imagesc(self.imdata-1./feval(sfun,XI,YI))
            hold on; plot(cols2,locs,'wx');hold off
            title('PIMMS Echelle Detector Image')

            %self.imdata=self.imdata-1./feval(sfun,XI,YI)
\end{verbatim}
\begin{verbatim}
            imagesc(log10(self.imdata))

            return
            self.forceTrace=true;
            self.forceExtract=true;

            self.getMaskForIncompleteOrders;
            self.traceSpectra;
            %self.specCenters=self.specCenters;
            self.extractSpectra;
            self.getP2PVariationsAndBlaze
            set(0,'DefaultFigureWindowStyle','docked')
\end{verbatim}
\begin{verbatim}
        end

        function spectraValues=extractSpectra(self)
            % extract spectra using trace - each order done individually (faster).

            if ~exist(self.SpectraFitsSaveFileName,'file') || self.forceExtract
                spectraValues=zeros(self.numOfFibers,self.imdim(2),self.numOfOrders);
                spectraVar=zeros(self.numOfFibers,self.imdim(2),self.numOfOrders);
                backgroundValues=zeros(size(self.imdata));self.imdata;

                assertWarn(self.forceExtract,...
                    'MISPRINT:extractSpectra:forceExtractFlagSet',...
                    'Force extraction flag set, starting extraction. Data will be overwritten')
                RN=self.readNoise/self.gain;
                for order=1:self.numOfOrders
                    spectra=zeros(self.numOfFibers,self.imdim(2));
                    specVar=zeros(self.numOfFibers,self.imdim(2));
                    %background=zeros(size(self.imdata));

                    %disp(['Extracting Order: ' num2str(order)])
                    orderSpecCenters=shiftdim(self.specCenters(order,:,:),1); % clips order dim (1,f,col) > (f,col)

                    % split into apetures
                    for col=1:self.imdim(2)

                        orderCenter=mean(orderSpecCenters(:,col));
                        profileApeture{col}=max(round(orderCenter-self.meanOrderWidth/2),1):...
                            min(round(orderCenter+self.meanOrderWidth/2),self.imdim(1));

                        orderProfile{col}=self.imdata(profileApeture{col},col)';
                        varProfile{col}=self.imvariance(profileApeture{col},col)';

                    end

                    % do extraction
                    %                     for col=1:self.imdim(2)
                    %                         [spectra(:,col), specVar(:,col), background(col,:)]=self.(self.OXmethod)(...
                    %                             profileApeture(col,:),orderProfile(col,:),varProfile(col,:),...
                    %                             squeeze(self.specCenters(order,:,col))',...
                    %                             squeeze(2*log(2)*self.specWidth(order,:,col))',...
                    %                             self.readNoise/self.gain);
                    %                     end

                    [spectra, specVar, background]=self.MPDoptimalExt(...
                        profileApeture,orderProfile,varProfile,...
                        (shiftdim(self.specCenters(order,:,:),1))',...
                        2*log(2)*(shiftdim(self.specWidth(order,:,:),1))',...
                        RN);



                    % unfold into final variables
                    for col=1:self.imdim(2)
                        backgroundValues(profileApeture{col},col)=background{col};
                    end
                    spectraValues(:,:,order)=spectra;
                    spectraVar(:,:,order)=specVar;
                end

                spectra1DHDR=createcards('NUMORDER',self.numOfOrders,'number of orders');
                spectra1DHDR.addcard('NUMFIBER',self.numOfFibers,'number of fibers')
                spectra1DHDR.addcard('TRACE',self.spectraTracePath,' ')

                fitswrite(spectraValues,self.SpectraFitsSaveFileName,spectra1DHDR.cards)
                fitswrite(spectraVar,self.SpectraFitsSaveFileName,'writemode','append')
                %fitswrite(backgroundValues,self.SpectraFitsSaveFileName,'writemode','append')
            else
                disp(['Pre-existing extraction data found at: ' self.SpectraFitsSaveFileName])

                spectraValues=fitsread(self.SpectraFitsSaveFileName);
                spectraVar=fitsread(self.SpectraFitsSaveFileName,'image',1);
                %backgroundValues=fitsread(self.SpectraFitsSaveFileName,'image',2);
            end
            self.spectraValues=spectraValues;
            self.spectraVar=spectraVar;
            %self.backgroundValues=backgroundValues;
        end

        function [spectraValues, spectraErrors, background, chi2]=MPDoptimalExtBack(self,dataRows,orderProfile,varProfile,specCenters,specWidth,RN)
            % Multi-Profile Deconvolution Optimal Extraction as described by Sharp & Birchall (2010)
            %
            % paper: Sharp R., Birchall M. N. (2010) Optimal Extraction of Fibre Optic Spectroscopy. PASA 27, pp. 91-103.
            %        http://dx.doi.org/10.1071/AS08001


            if iscolumn(orderProfile); orderProfile=orderProfile'; disp(1); end
            if iscolumn(varProfile); varProfile=varProfile'; disp(2); end
            if iscolumn(dataRows); dataRows=dataRows'; disp(3); end
            if isrow(specCenters);specCenters=specCenters'; disp(4); end
            if isrow(specWidth);specWidth=specWidth'; disp(5); end
            %            error(' ')
            %            save('testing.mat','self','dataRows','orderProfile','varProfile','specCenters','specWidth','RN')

            phi=self.getPhi(dataRows,specCenters,2*log(2)*specWidth,[ones(length(specCenters),1)]);

            [xout,~,~,~] = fminsearch(@optimizeBackgroundFit, polyfit(dataRows,orderProfile/sum(orderProfile),1));

            [chi2, fittedValues, fittedErrors, M]=optimizeBackgroundFit(xout);

            spectraValues=fittedValues(1:end-1);
            spectraErrors=fittedErrors(1:end-1);
            background=fittedValues(end)*polyval(xout,dataRows)/sum(polyval(xout,dataRows));

            function [chi2, fittedValues, fittedErrors, M]=optimizeBackgroundFit(x)
                %setup
                phifit=[phi; polyval(x,dataRows)/sum(polyval(x,dataRows))];%ones(1,size(phi,2))/size(phi,2) %([1:size(phi,2)]*x(1)+x(2)) / sum([1:size(phi,2)]*x(1)+x(2))
                sigmaweightedPhi=bsxfun(@rdivide,phifit,sqrt(varProfile))';
                c=phifit*sigmaweightedPhi;
                b=((orderProfile)*sigmaweightedPhi)';

                %setup error
                ce=phifit*phifit';
                be=((varProfile-RN^2)*phifit')';

                %solve
                fittedValues=c\b;
                fittedErrors=ce\be;

                %Model
                M=sum(bsxfun(@times,phifit,fittedValues),1);

                chi2=sum(((orderProfile-M)).^2./varProfile)/(size(M,2)-size(fittedValues,1)-length(x));
                %                 if chi2>1
                %                     plot(1:195,M,1:195,orderProfile)
                %                     drawnow;
                %                 end
            end
        end

        function [spectraValues, spectraErrors, background]=MPDoptimalExt(self,dataRows,orderProfile,varProfile,specCenters,specWidth,RN)
            % Multi-Profile Deconvolution Optimal Extraction as described by Sharp & Birchall (2010)
            %
            % paper: Sharp R., Birchall M. N. (2010) Optimal Extraction of Fibre Optic Spectroscopy. PASA 27, pp. 91-103.
            %        http://dx.doi.org/10.1071/AS08001

            %setup
            %             if 0
            %                 phi=self.getPhi(dataRows,specCenters,specWidth,ones(length(specCenters),1));
            %             %%phi
            %             else
            %             phi1=;
            %             phi2=;
            %             phi3=
            spectraValues=zeros(size(specCenters'));
            spectraErrors=spectraValues;
            for col=1:size(specCenters,1)
                phi=bsxfun(@times, exp(-(bsxfun(@rdivide, bsxfun(@minus,repmat(dataRows{col},...
                    [size(specCenters,2),1]),specCenters(col,:)'), specWidth(col,:)')).^2), 1./(specWidth(col,:)'*sqrt(pi)));
                %phi=bsxfun(@times, phi4, specPeaks);
                %phi=sparse(phi);
                phi(phi<1e-6)=0;
                %             end

                %            if 1
                sigmaweightedPhi=bsxfun(@rdivide,phi,sqrt(varProfile{col}))';
                c=phi*sigmaweightedPhi;
                b=((orderProfile{col})*sigmaweightedPhi)';
                %             else
                %                 sigmaweightedPhi=bsxfun(@rdivide,phi,sqrt(varProfile))';
                %                 c=mtimesx(phi,sigmaweightedPhi,'MATLAB');
                %                 b=mtimesx(orderProfile,sigmaweightedPhi,'MATLAB')';
                %             end

                %setup error
                ce=phi*phi';
                be=((varProfile{col}-RN^2)*phi')';

                %solve
                spectraValues(:,col)=(c\b);
                %spectraValues=linsolve(c,b);
                spectraErrors(:,col)=(ce\be);
                %spectraErrors=linsolve(ce,be);

            end
            background=cellfun(@(x) zeros(size(x)),orderProfile,'UniformOutput',false);

        end

        function [spectraValues, spectraErrors, background]=MPDoptimalExtOld(self,dataRows,orderProfile,varProfile,specCenters,specWidth,RN)
            % Multi-Profile Deconvolution Optimal Extraction as described by Sharp & Birchall (2010)
            %
            % paper: Sharp R., Birchall M. N. (2010) Optimal Extraction of Fibre Optic Spectroscopy. PASA 27, pp. 91-103.
            %        http://dx.doi.org/10.1071/AS08001

            %setup
            %             if 0
            %                 phi=self.getPhi(dataRows,specCenters,specWidth,ones(length(specCenters),1));
            %             %%phi
            %             else
            phi1=bsxfun(@minus,repmat(dataRows,[length(specCenters),1]),specCenters);
            phi2=bsxfun(@rdivide, phi1, specWidth);
            phi3=exp(-(phi2).^2);
            phi=bsxfun(@times, phi3, 1./(specWidth*sqrt(pi)));
            %phi=bsxfun(@times, phi4, specPeaks);
            %phi=sparse(phi);
            phi(phi<1e-8)=0;
            %             end

            %            if 1
            sigmaweightedPhi=bsxfun(@rdivide,phi,sqrt(varProfile))';
            c=phi*sigmaweightedPhi;
            b=((orderProfile)*sigmaweightedPhi)';
            %             else
            %                 sigmaweightedPhi=bsxfun(@rdivide,phi,sqrt(varProfile))';
            %                 c=mtimesx(phi,sigmaweightedPhi,'MATLAB');
            %                 b=mtimesx(orderProfile,sigmaweightedPhi,'MATLAB')';
            %             end

            %setup error
            ce=phi*phi';
            be=((varProfile-RN^2)*phi')';

            %solve
            spectraValues=(c\b);
            %spectraValues=linsolve(c,b);
            spectraErrors=(ce\be);
            %spectraErrors=linsolve(ce,be);
            background=zeros(size(orderProfile));

        end

        function phi=getPhi(~,dataRows,specCenters,specWidth,specPeaks)
            phi1=bsxfun(@minus,repmat(dataRows,[length(specCenters),1]),specCenters);
            phi2=bsxfun(@rdivide, phi1, specWidth);
            phi3=exp(-(phi2).^2);
            phi4=bsxfun(@times, phi3, 1./(specWidth*sqrt(pi)));
            phi=bsxfun(@times, phi4, specPeaks);
            %phi=sparse(phi);
            phi(phi<1e-8)=0;
        end

        function lineariseAndCombineSpectrum(self,saveFiles)

            if nargin==1
                saveFiles=false;
            end

            spec=(self.spectraValues);%./self.P2PVariationValues;%./self.flatBlaze;%
            specVar=(self.spectraVar);%./self.P2PVariationValues;%./self.flatBlaze;%

            %             for or=1:self.numOfOrders
            %                 specVar(:,:,or) = bsxfun(@rdivide,specVar(:,:,or),max(spec(:,:,or)')');
            %                 spec(:,:,or)    = bsxfun(@rdivide,spec(:,:,or),max(spec(:,:,or)')');
            %             end

            longwavelinear=linspace(min(self.wavefit(:)),max(self.wavefit(:)),self.imdim(2)*self.numOfOrders);

            speclinearlong=zeros(self.numOfFibers,self.imdim(2)*self.numOfOrders,self.numOfOrders);
            spectraVarlinearlong=zeros(self.numOfFibers,self.imdim(2)*self.numOfOrders,self.numOfOrders);

            for o=1:self.numOfOrders
                for f=1:size(spec,1);
                    speclinearlong(f,:,o)=interp1(self.wavefit(f,:,o),spec(f,:,o),longwavelinear,'spline',0);
                    spectraVarlinearlong(f,:,o)=interp1(self.wavefit(f,:,o),specVar(f,:,o),longwavelinear,'spline',0);
                    specflatlong(f,:,o)=interp1(self.wavefit(f,:,o),self.flatBlaze(f,:,o),longwavelinear,'spline',0);
                end
            end
            %specflatlong=ones(size(speclinearlong));

            finalspeclong=nansum(speclinearlong,3)';%./nansum(specflatlong,3)';
            finalspecVarlong=nansum(spectraVarlinearlong,3)';%./nansum(specflatlong,3)';
            flatspeclong=nansum(specflatlong,3)';

            finalspeclong=finalspeclong./bsxfun(@rdivide,flatspeclong,mean(flatspeclong));
            finalspecVarlong=finalspecVarlong./bsxfun(@rdivide,flatspeclong,mean(flatspeclong));

            toclip=isnan(sum(finalspeclong,2));

            longwavelinear_clipped=longwavelinear(~toclip);
            finalspecVarlong_clipped=finalspecVarlong(~toclip,:);
            finalspeclong_clipped=finalspeclong(~toclip,:);

            self.finalSpectra=squeeze(finalspeclong_clipped');
            self.finalSpectraVar=squeeze(finalspecVarlong_clipped');
            self.finalWave=longwavelinear_clipped;

            self.finalSpec=squeeze(sum(finalspeclong_clipped,2)');
            self.finalSpecVar=squeeze(sum(finalspecVarlong_clipped,2)');

            if 0
\end{verbatim}
\begin{verbatim}
                for i=1:size(finalspecVarlong_clipped,2)
                    smoother(:,i)=csaps(self.finalWave,finalspecVarlong_clipped(:,i),0.0000001,self.finalWave);
                end
\end{verbatim}
\begin{verbatim}
                smoother=mean(smoother,2)';
                %[smoother] = blazeCorrection(self.finalSpec,self.finalWave,0.98)';
\end{verbatim}
\begin{verbatim}
            else
                smoother=1;
            end
            %            error(' ')
            self.finalSpec=self.finalSpec./smoother;
            self.finalSpecVar=self.finalSpecVar./smoother;

            if saveFiles
                header=self.targetHeader;
                header.IMAGETYP='SPECTRUM';
                header.CRPIX1=round(length(self.finalWave)/2);
                header.CRVAL1=self.finalWave(header.CRPIX1);
                header.CTYPE1='Wavelength';
                header.CUNIT1='nm';
                header.CDELT1=mean(diff(self.finalWave));
                header.UTC=round((header.JD-floor(header.JD))*24*60*60);
                header.MJD=header.JD-2400000.5;
                header.DLAT=-33.873651000000000000;
                header.DLONG=151.206889600000070000;%sydney
                header.GEOELV=100;

                headercell1=fitstructure2cell(header);

                header2.EXTNAME='FLUXERROR';
                headercell2=fitstructure2cell(header2);

                fitswrite(finalspeclong_clipped,[self.targetBaseFilename '-IndivCalSpec.fit'],'keywords',headercell1(8:end,:))
                fitswrite(finalspecVarlong_clipped,[self.targetBaseFilename '-IndivCalSpec.fit'],'writemode','append','keywords',headercell2)

                fitswrite(self.finalSpec,[self.targetBaseFilename '-CombCalSpec.fit'],'keywords',headercell1(8:end,:))
                fitswrite(self.finalSpecVar,[self.targetBaseFilename '-CombCalSpec.fit'],'writemode','append','keywords',headercell2)
            end
        end
    end

    methods (Static, Access = private)
        function answer=checkForReducedFitsAt(path)
            % check for a reduced target
            try
                import matlab.io.*
                fptr = fits.openFile(path);
                fits.closeFile(fptr);
                answer=1;
            catch err
                if strcmp(err.identifier,'MATLAB:imagesci:fits:libraryError')
                    error('MISPRINT:checkForReducedTarget:fitsOpenError','Reduced fits file does not exist.')
                else
                    rethrow(err)
                end
            end
        end
    end
    methods (Static)
        [specCenters, p, mu]=polyfitwork(imdim,means,column,polyorder,offset,plotalot)
        prepareFrames
        [peaks,means,widths,xfitted] = fitNGaussainsAlt(N,x,y,peakcut,plotting)
        out=nGausFunc(x,xData,N)
        wavecalGUI
        autoimprovewavelength(varargin)
    end
end
\end{verbatim}



%\end{document}
    
